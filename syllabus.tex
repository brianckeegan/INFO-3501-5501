\documentclass[11pt]{memoir}

% based on kieran healy's memoir modifications
\usepackage{mako-mem}
\chapterstyle{article-2}
\pagestyle{mako-mem}

\usepackage{ucs}
\usepackage[utf8x]{inputenc}

%\usepackage{kpfonts}
%\usepackage[bitstream-charter]{mathdesign}
\usepackage{fbb}
\usepackage[T1]{fontenc}
%\usepackage{textcomp}

%\renewcommand{\rmdefault}{ugm}
%\renewcommand{\sfdefault}{phv}
%

\usepackage[letterpaper,left=1.25in,right=1.25in,top=1.25in,bottom=1.25in]{geometry}

% packages i use in essentially every document
\usepackage{graphicx}
\usepackage{enumerate}

% Setup list environments
\usepackage{enumitem}
\setlist[description]{
  topsep=0pt,
  before=\vspace{5pt},
  after=\vspace{0pt},
  itemsep=2pt,
  labelsep=0pt
}

% packages i use in many documents but leave off by default
% \usepackage{amsmath, amsthm, amssymb}
% \usepackage{dcolumn}
% \usepackage{endfloat}

% Set paragraph indents and spacing
\setlength{\parindent}{0pt}
\setlength{\parskip}{.5\baselineskip}
%\usepackage[document]{ragged2e}

% adjust section title formatting
\usepackage{titlesec}
\titlespacing\section{-8pt}{8pt plus 2pt minus 2pt}{-6pt}
\titlespacing\subsection{-8pt}{8pt plus 2pt minus 2pt}{-6pt}
\titlespacing\subsubsection{0pt}{8pt plus 2pt minus 2pt}{10pt}

% allows full, in-line citations
\usepackage{bibentry} 

% add bibliographic stuff 
\usepackage[round, numbers]{natbib} \def\citepos#1{\citeauthor{#1}'s (\citeyear{#1})} \def\citespos#1{\citeauthor{#1}' (\citeyear{#1})}
\renewcommand{\bibnumfmt}[1]{}

% define colors from http://www.colorado.edu/brand/visual-identity/typography-color
\usepackage[usenames,dvipsnames]{color}
\definecolor{CUGold}{RGB}{207,184,124}
\definecolor{CUDarkGray}{RGB}{86,90,92}
\definecolor{CULightGray}{RGB}{162,164,163}

% customize URLs
%\usepackage{url}
%\usepackage{breakurl} 
\usepackage[breaklinks, bookmarks, bookmarksopen]{hyperref}
\hypersetup{colorlinks=true, linkcolor=Blue, citecolor=Black, filecolor=Blue,
    urlcolor=Blue, unicode=true, breaklinks=true}

% create a "reading list" environment to format the items
\newenvironment{readinglist}{
\begin{list}{}{\leftmargin=8pt \itemindent=0em}
  \setlength{\itemsep}{8pt}
  \setlength{\parskip}{0em}
  \setlength{\parsep}{1em}
  \setlength{\parindent}{8em}}
{\end{list}}

% Course/Instructor metadata -- alter as neded
\def\myauthor{Brian Keegan}
\def\mycoursename{Peer Production and Crowdsourcing}
\def\mycourselisting{INFO 3501/5501}
\def\myoffice{ENVD 201}
\def\myclassroom{HUMN 190}
\def\mymeetingtime{Monday \& Wednesday 17:30--18:45}
\def\mydate{Fall 2016}
\def\myemail{brian.keegan@colorado.edu}
\def\myweb{http://www.brianckeegan.org}
\def\myofficehours{Monday 15:00--17:00 or by appointment}

% Some that I'm not using here:
\def\mytitle{Assistant Professor }
\def\mycopyright{\myauthor}
\def\myphone{(+1) 617-803-6971}
\def\mystreet{1060 18th St.}
\def\mycity{Boulder, CO 80302}

\begin{document}

\nobibliography*

%\baselineskip 14.2pt

\title{
    \textit{\normalsize{\textcolor{CUGold}{\textbf{Problems in Information Science:}}}}\\
    \textbf{\huge{\mycoursename}}\\
    \vspace{5pt} \normalsize{\mycourselisting}
    }

\author{\mydate\\ \mymeetingtime\\ \myclassroom}

\date{\normalsize{\mytitle \myauthor\\
       E-mail: \href{mailto:\myemail}{\myemail}\\
       Office: \myoffice\\
       Office hours: \myofficehours}}

\maketitle

%%%%%%%%%%%%%%%%%%%%%%
%% Acknowledgements %%
%%%%%%%%%%%%%%%%%%%%%%
% This syllabus template was made in LaTeX by Brian Keegan and is distributed as Free Software under the GNU GPL v3. It was built using style templates created by Aaron Shaw, Benjamin Mako Hill, and Kieran Healy.

\section{\textbf{Course Description}}

This is a research seminar that will analyze the social and technical mechanisms that enable popular peer production and crowdsourcing systems like Wikipedia. Understanding how social processes and technical structures intersect to enable new kinds of interaction is a central question within information science. How does the design of a peer production system tap into basic social, psychological, and organizational processes? What kinds of behavioral data can researcher extract and analyze from these systems? How can the success or failure of these systems inform the design of alternative economic and political models? This course is part of the ``Problems in Information Science'' series, which brings contemporary research to the classroom in the form of project-based investigation.

\subsection{Learning Objectives and Course Design}

By the end of the semester, students should be able to: (1) discuss and compare social processes occurring across different peer production and crowdsourcing systems; (2) identify and collect data generated by these systems; (3) design exploratory research using these data and interpret their findings; and (4) propose and evaluate the feasibility of peer production and crowdsourcing designs to expand to other social, cultural, economic, and political spheres.

The course consists of three, five-week units. The first unit focuses on peer production and crowdsourcing as \textit{social} systems by exploring how motivations, norms, and governance interact. The second unit focuses on peer production and crowdsourcing as \textit{technical} systems by exploring the systems that structure user interactions and the kinds of data they record. The third unit focuses on peer production and crowdsourcing as \textit{alternative} systems to prevailing models for decision making, collaboration, and economic production.

Class will meet twice per week on Monday and Wednesday from 17:30 to 18:45 (5:30pm to 6:45pm) in HUMN 190. The format of each class will vary between lecture, discussion, activities, and lab formats depending on the learning objectives of that week. Students will complete three kinds of assignments over the course of the class: (1) weekly reflections, (2) lab exercises, and (3) a final project. The final project will be developed in stages over the course of the semester with incremental contributions and peer feedback.

\subsection{Prerequisites}
There are no formal prerequisites for this course. However, you should have some familiarity (prior class, research project, internship experience, MOOC coursework, \textit{etc.}) with methods for data mining (web scraping, information retrieval from APIs, \textit{etc}.) and quantitative analysis (statistical inference, machine learning, \textit{etc}.). Knowledge of statistical software or scientific computing languages such as \texttt{R} or \texttt{Python} will also be helpful, but is not required. If you have questions or concerns about these prerequisites, please \href{mailto:brian.keegan@colorado.edu}{email me}.

\subsection{Requirements}
As a research seminar, students' regular and sustained participation in all class activities is essential. Attending and participating in all seminar meetings, completing all assignments, providing feedback on peers' contributions are all required from every student. If you need to be excused from attendance or need an extension, please \href{mailto:brian.keegan@colorado.edu}{notify me via email} at least 24 hours in advance.

\subsection{Course Website and Materials}
Materials for the class (including a ``live'' version of the course schedule, readings, discussion board, lab assignments, tutorials, and data) will be made available through \mbox{Desire2Learn}:
\vspace{-8pt}
    \begin{center}
    \href{https://https://learn.colorado.edu/d2l/home/179594}{https://learn.colorado.edu/d2l/home/179594}
    \end{center}
\vspace{-8pt}
Once the semester begins, this PDF version of the syllabus will be revised infrequently and any revised requirements will be posted as announcements and updated course schedule to \mbox{Desire2Learn}.

If you are interested in understanding Wikipedia in greater depth, I would recommend the following books:

\begin{readinglist} 
    \item \bibentry{ayers_wikipedia_2008}
    \item \bibentry{benkler_wealth_2006}
    \item \bibentry{jemielniak_common_2014}
    \item \bibentry{kraut_building_2012}
    \item \bibentry{reagle_good_2010}
\end{readinglist}

\subsection{Assignments}
There are three kinds of assignments in the class: (1) reading responses, (2) lab assignments, and a (3) final project. Assignments should be submitted to me via \mbox{Desire2Learn}. I strongly prefer submissions to be in PDF format to maximize compatibility and reduce security risks. Reading responses should be submitted before 9:00 on their due dates for me to review them before class. Lab assignments should include your well-commented code documenting your process as well as the final outputs. Students will complete and submit their work individually but if you work with others, acknowledge your collaborators by name on the submission. You should be prepared to explain your implementation in a 1-on-1 code review if I request it.

\subsubsection{Readings}
There is no textbook for the class and we will read magazine articles, academic papers, or book chapters. PDFs of these materials will be available on the \mbox{Desire2Learn} class site. For many of the class sessions I have also included a sub-section of ``Additional Readings'' that are purely optional but may be useful if students are interested in reading in further depth to support a research project, preparing for prelims, \textit{etc}.

\subsubsection{Lab Assignments}

\subsubsection{Final Project}

\subsection{Statistical Computing}
You will need to use statistical computing software in this class. You are welcome to use any scripting or statistical software you like (Python, \texttt{R}, \textit{etc}.); however, I will use \href{http://jupyter.org/}{Jupyter notebooks} written in Python for all in-class examples and solutions to the problem sets. I recommend using the \href{https://www.continuum.io/why-anaconda}{Anaconda distribution} of Python that includes the vast majority of libraries and functionality to support data collection and analysis for this class. You are however welcome to use any statistical software with which you are comfortable to complete lab assignments.

\subsection{Evaluation} 
Your final grade for the course will be based on my evaluation of the three kinds of assignments as well as your general participation within discussions during class and online.

    \begin{description}[itemsep=0pt,labelsep=0pt]
        \item[Reading responses]~($30\%$)
        \item[Lab assignments]~($30\%$)
        \item[Final project]~($25\%$)
        \item[Participation]~($15\%$)
    \end{description}

\section{Course Policies}

\subsection{In-Class Confidentiality}
The success of this class depends on participants feeling comfortable sharing questions, ideas, concerns, and confusions about works-in-progress, the research process, and their personal experiences during class (in-class discussion, online discussions, and written assignments). These assignments and discussions should be considered confidential and generally should not be discussed with people outside of class. You may read and comment on classmates' writing, code, and images for the sole purpose of use within this class. You may not use, run, copy, perform, display, distribute, modify, translate, or create derivative works of other students' work outside of this class without their expressed written consent or formal license. Furthermore, you may not create any audio, video, or other records during class time nor may you share comments attributable to other people's identities without getting that person's permission.

\subsection{Faculty Interaction}
In addition to teaching this class, I also (1) manage a research program; (2) advise students; (3) perform service for the academic community; and (4) live my life as a private citizen. I will check e-mail between 8:00 and 18:00 on non-holiday business days and try to respond within 24 hours. I welcome online or offline interactions outside of class, however these are not appropriate spaces for discussing class matters. \href{maito:brian.keegan@colorado.edu}{E-mailing me}, coming to my office hours, or scheduling an appointment are the best ways to ask questions, discuss concerns, or get feedback outside of class.

\subsection{Research Ethics and Professional Conduct}
This course will involve participating in online communities, analyzing data available on the web, and prototyping alternative peer production and crowdsourcing systems. At all times you should make sure that your actions simultaneously minimize the risk of harm to other people as well to yourself. You should follow the Terms of Service and Privacy Policy when collecting data from websites; only collect publicly-accessible data; identify yourself as a student or researcher when interacting with others; and allow participants to remove themselves from your study if they request it. More details about research ethics with online data can be found in the Association for Computing Machinery's \href{https://www.acm.org/about-acm/acm-code-of-ethics-and-professional-conduct}{Code of Ethics and Professional Conduct}, Association for Internet Researchers' \href{http://aoir.org/ethics/}{ethics working committee reports}, American Psychological Association's \href{http://www.apa.org/science/leadership/bsa/internet/internet-report.aspx}{online research report}.

\subsection{Deadlines and Absences}
If something causes you to miss a deadline or a class, please contact me. If you request --- and obtain --- an incomplete for the course and/or an extension on the final project (note: I \textit{strongly} discourage this!), please allow at least 4 weeks after you submit your completed work for me to submit a grade. Keep this in mind if you will need the grade in order to receive your fellowship/diploma/visa/\textit{etc}. by a particular date.

\subsection{Accommodations for Disabilities}
I am committed to providing everyone the support and services needed to participate in this course. If you qualify for accommodations because of a disability, please submit to your professor a letter from Disability Services in a timely manner (for exam accommodations provide your letter at least one week prior to the exam) so that your needs can be addressed. Disability Services determines accommodations based on documented disabilities. Contact Disability Services at 303-492-8671 or by e-mail at \href{mailto:dsinfo@colorado.edu}{dsinfo@colorado.edu}. If you have a temporary medical condition or injury, see \href{http://www.colorado.edu/disabilityservices/students/temporary-medical-conditions}{Temporary Medical Conditions} guidelines under Quick Links at \href{http://www.colorado.edu/disabilityservices/students}{Disability Services} website and discuss your needs with me.

\subsection{Religious Observance}
Campus policy regarding \href{http://www.colorado.edu/policies/observance-religious-holidays-and-absences-classes-andor-exams}{religious observances} requires that faculty make every effort to deal reasonably and fairly with all students who, because of religious obligations, have conflicts with scheduled exams, assignments or required assignments/attendance. If this applies to you, please email me directly as soon as possible at the beginning of the term. 

\subsection{Classroom Behavior}
Students and faculty each have responsibility for maintaining an appropriate learning environment. Those who fail to adhere to such behavioral standards may be subject to discipline. Professional courtesy and sensitivity are especially important with respect to individuals and topics dealing with differences of race, color, culture, religion, creed, politics, veteran’s status, sexual orientation, gender, gender identity and gender expression, age, ability, and nationality.  Class rosters are provided to the instructor with the student's legal name. I will gladly honor your request to address you by an alternate name or gender pronoun. Please advise me of this preference early in the semester so that I may make appropriate changes. For more information, see the policies on \href{http://www.colorado.edu/policies/student-classroom-and-course-related-behavior}{class behavior} and the \href{http://www.colorado.edu/osc/#student_code}{student code}.

\subsection{Harassment and Discrimination}
The University of Colorado Boulder (CU Boulder) is committed to maintaining a positive learning, working, and living environment. CU Boulder will not tolerate acts of sexual misconduct, discrimination, harassment or related retaliation against or by any employee or student. CU's \href{http://www.colorado.edu/policies/discrimination-and-harassment-policy-and-procedures}{Sexual Misconduct Policy} prohibits sexual assault, sexual exploitation, sexual harassment, intimate partner abuse (dating or domestic violence), stalking or related retaliation. CU Boulder's \href{http://www.colorado.edu/policies/discrimination-and-harassment-policy-and-procedures}{Discrimination and Harassment Policy} prohibits discrimination, harassment or related retaliation based on race, color, national origin, sex, pregnancy, age, disability, creed, religion, sexual orientation, gender identity, gender expression, veteran status, political affiliation or political philosophy. Individuals who believe they have been subject to misconduct under either policy should contact the Office of Institutional Equity and Compliance (OIEC) at 303-492-2127. Information about the OIEC, the above referenced policies, and the campus resources available to assist individuals regarding sexual misconduct, discrimination, harassment or related retaliation can be found at the \href{http://www.colorado.edu/institutionalequity/}{OIEC website}.

\subsection{Honor Code}
All students enrolled in a University of Colorado Boulder course are responsible for knowing and adhering to the \href{http://www.colorado.edu/policies/academic-integrity-policy}{academic integrity policy} of the institution. Violations of the policy may include: plagiarism, cheating, fabrication, lying, bribery, threat, unauthorized access, clicker fraud, resubmission, and aiding academic dishonesty. All incidents of academic misconduct will be reported to the Honor Code Council (\href{mailto:honor@colorado.edu}{honor@colorado.edu}; 303-735-2273). Students who are found responsible for violating the academic integrity policy will be subject to nonacademic sanctions from the Honor Code Council as well as academic sanctions from the faculty member. Additional information regarding the academic integrity policy can be found at \href{http://honorcode.colorado.edu}{honorcode.colorado.edu}. 

\section{Acknowledgements}
The design and format of this course borrows from similar courses offered by other professors whose research and thinking has been very influential on me: \href{http://aaronshaw.org/}{Aaron Shaw}'s ``\href{http://aaronshaw.org/teaching/2014/occ/syllabus/Shaw-occ-syllabus2014.pdf}{Online Communities \& Crowds}'' at Northwestern University's School of Communication, \href{https://mako.cc/academic/}{Benjamin Mako Hill}'s ``\href{http://wiki.communitydata.cc/Interpersonal_Media_(Fall_2015)}{Interpersonal Media}'' at the University of Washington's Department of Communication, and \href{http://reagle.org/joseph}{Joseph Reagle}'s ``\href{http://reagle.org/joseph/2014/oc/oc-syllabus-FA.html}{Online Communities}'' at Northeastern University's Department of Communication Studies.

%%%%%%%%%%%%%%%%%%%%%%
%%%%%%%%%%%%%%%%%%%%%%
%%% COURSE OUTLINE %%%
%%%%%%%%%%%%%%%%%%%%%%
%%%%%%%%%%%%%%%%%%%%%%

\newpage
\section{\textbf{Course Outline}}

As described above, all readings and assigned tasks are due
\emph{prior} to the class alongside which they are listed. Items
listed in the ``Additional Readings'' subsection for each week are not
required. I may adjust the list of readings or the schedule as needed
throughout the quarter, so please consult the online version of the
schedule for the most up-to-date information. We will meet in class a total of 29 times over the course of the the 16-week semester.

%%%%%%%%%%%%%
%% WEEK 01 %%
%%%%%%%%%%%%%

% Confirm research computing environments for Python and \texttt{R} are properly configured and ready for data collection and analysis. 

\section{0 -- Introductions \& Research Ethics}
\textcolor{CUGold}{\textbf{Monday, August 22}}\\
Review syllabus, learning objectives, and course policies. Introduce yourselves and share motivations and goals for the class. Discuss ethical considerations when conducting online research such as collecting ``public'' data, following Terms of Service, and protecting human subjects.

    \subsection{Readings}
    \begin{readinglist}
        \item This syllabus.
        \item \bibentry{bruckman_teaching_2006}
        \item \bibentry{zimmmer_okcupid_2016}
        \item \bibentry{wiki_terms_of_use}
    \end{readinglist}

%\subsection{Additional readings}

\section{1 -- A Wager}
\textcolor{CUGold}{\textbf{Wednesday, August 24}}\\
In 2006, Nick Carr and Yochai Benker made a wager about the future of online information production. More than a decade later, who --- if anyone --- won? We will develop preliminary definitions of peer production and crowdsourcing and assess their fit to the current landscape.

    \subsection{Readings}
    \begin{readinglist}
        \item \bibentry{carr_calacaniss_2006a}
        \item \bibentry{carr_benkler_2006b}
        \item \bibentry{carr_pay_2012}
        \item \bibentry{benkler_carr-benkler_2012}
    \end{readinglist}

    \subsection{Assignments}
    \begin{description}%[itemsep=2pt,labelsep=0pt]
        \item[Reading Response 1 ] -- due Wednesday, August 24. ``Will the decentralized web collectives be able to operate successfully outside `the price system' and without `managerial structure'?'' Do the most influential web platforms in 2016 follow peer production processes or are they price-incentivized systems? Given how web platforms now operate, what considerations did Carr and Benkler overlook?
    \end{description}

    \subsection{Additional Readings}
    \begin{readinglist}
        \item \bibentry{benkler_ch3_2006}
        \item \bibentry{carr_amorality_2005}
        \item \bibentry{keen_web_2006}
        \item \bibentry{lanier_digital_2006}
    \end{readinglist}

%%%%%%%%%%%%%
%% WEEK 02 %%
%%%%%%%%%%%%%
\section{2 -- Motivation}
\textcolor{CUGold}{\textbf{Monday, August 29}}\\
What motivates participants to join and contribute to online communities? We will examine how different motivations contribute to or detract from the success of peer production systems. 

    \subsection{Readings}
    \begin{readinglist}
        \item \bibentry{wikipedia_why_2016}
        \item \bibentry{kraut_building_2012}. Chapter 2.
        \item \bibentry{nov_what_2007}
    \end{readinglist}
    
    \subsection{Additional Readings}
    \begin{readinglist}
        \item \bibentry{kuznetsov_motivations_2006}
        \item \bibentry{lakhani_why_2003}
        \item \bibentry{lampe_motivations_2010}
        \item \bibentry{roberts_understanding_2006}
        \item \bibentry{rafaeli_online_2008}
        \item \bibentry{yang_motivations_2010}
    \end{readinglist}

\section{3 -- Commitment}
\textcolor{CUGold}{\textbf{Wednesday, August 31}}\\
Why do participants stick around and become active contributors within online communities? We discuss how the design impacts user commitment as well as a framework explaining how users transition into leadership roles.

    \subsection{Readings}
    \begin{readinglist}
        \item \bibentry{wikipedia_wikifauna_2016}
        \item \bibentry{kraut_building_2012}. Chapter 3.
        \item \bibentry{preece_readerleader_2009}
    \end{readinglist}
    
    \subsection{Assignments}
    \begin{description}%[itemsep=2pt,labelsep=0pt]
        \item[Reading Response 2 ] -- due Wednesday, August 31. 
    \end{description}
    
    \subsection{Additional Readings}
    \begin{readinglist}
        \item \bibentry{panciera_wikipedians_2009}.
        \item \bibentry{}
        item \bibentry{}
    \end{readinglist}

%%%%%%%%%%%%%
%% WEEK 03 %%
%%%%%%%%%%%%%
\section{4 -- Labor Day}
\textcolor{CUGold}{\textbf{Monday, September 5 --- NO CLASS}}\\

\section{5 -- Newcomers}
\textcolor{CUGold}{\textbf{Wednesday, September 7}}\\

    \subsection{Readings}
    \begin{readinglist}
        \item \bibentry{wikipedia_seven_2015}
        \item \bibentry{halfaker_dont_2011}
        \item \bibentry{morgan_tea_2013}
    \end{readinglist}

%%%%%%%%%%%%%
%% WEEK 04 %%
%%%%%%%%%%%%%
\section{6 -- Culture and Norms}
\textcolor{CUGold}{\textbf{Monday, September 12}}\\

    \subsection{Readings}
    \begin{readinglist}
        \item \bibentry{wikiphilosophies}
        %\item Reagle, J. (2010). Good faith collaboration : the culture of Wikipedia. MIT Press, Cambridge Mass. Chapters 1 \& 3.
        \item \bibentry{bryant_becoming_2005}
        \item \bibentry{kriplean_articulations_2008}
    \end{readinglist}

    \subsection{Additional Readings}
    \begin{readinglist}
        \item \bibentry{burke_mopping_2008}
        \item \bibentry{danescu-niculescu-mizil_echoes_2012}
        \item \bibentry{thom_whats_2009}
        \item \bibentry{zhu_organizing_2012}
    \end{readinglist}

\section{7 -- Rules and Governance}
\textcolor{CUGold}{\textbf{Wednesday, September 14}}\\

    \subsection{Readings}
    \begin{readinglist}
        \item \bibentry{butler_dont_2008}
        \item \bibentry{shaw_laboratories_2014}
    \end{readinglist}
    
    \subsection{Assignments}
    \begin{description}%[itemsep=2pt,labelsep=0pt]
        \item[Reading Response 3 ] -- due Wednesday, September 14. 
    \end{description}
    
    \subsection{Additional Readings}
    \begin{readinglist}
        \item \bibentry{beschastnikh_wikipedian_2008}
        \item \bibentry{black_self-governance_2011}
        \item \bibentry{forte_decentralization_2009}
        \item \bibentry{kriplean_community_2007}
        \item \bibentry{kostakis_peer_2010}
        %\item \bibentry{konieczny_governance_2009}
        \item \bibentry{konieczny_adhocratic_2010}
        \item \bibentry{leskovec_governance_2010}
        \item \bibentry{omahony_emergence_2007}
    \end{readinglist}


%%%%%%%%%%%%%
%% WEEK 05 %%
%%%%%%%%%%%%%
\section{8 -- Conflict}
\textcolor{CUGold}{\textbf{Monday, September 19}}\\

    \subsection{Readings}
    \begin{readinglist}
        \item \bibentry{kittur_he_2007}
        \item \bibentry{joyce_handling_2011}
    \end{readinglist}

\section{9 -- Bias}
\textcolor{CUGold}{\textbf{Wednesday, September 21}}\\
How susceptible are peer production systems to the biases of their contributors? We discuss research findings about geographic, cultural, gender, and political biases in Wikipedia.

    \subsection{Readings}
    \begin{readinglist}
        \item \bibentry{hecht_measuring_2009}
        \item \bibentry{wagner_women_2016}
        \item \bibentry{greenstein_is_2012}
    \end{readinglist}
    
    \subsection{Assignments}
    \begin{description}%[itemsep=2pt,labelsep=0pt]
        \item[Reading Response 4 ] -- due Wednesday, September 21. 
    \end{description}
    
    \subsection{Additional Readings}
    \begin{readinglist}
        \item \bibentry{antin_gender_2011}
        \item \bibentry{collier_conflict_2012}
        \item \bibentry{graellsgarrido_first_2015}
        \item \bibentry{iosub_emotions_2014}
        \item \bibentry{johnson_not_2016}
        \item \bibentry{lam_wpclubhouse_2011}
        \item \bibentry{reagle_gender_2011}
    \end{readinglist}

%%%%%%%%%%%%%
%% WEEK 06 %%
%%%%%%%%%%%%%
\section{10 -- Revision Histories}
\textcolor{CUGold}{\textbf{Monday, September 26}}\\

    \subsection{Readings}
    \begin{readinglist}
        \item \bibentry{viegas_studying_2004}
        \item \bibentry{geiger_using_2013}
    \end{readinglist}
    
    \subsection{Additional Readings}
    \begin{readinglist}
        \item \bibentry{ekstrand_rv_2009}
    \end{readinglist}

\section{11 -- Revision Histories Lab}
\textcolor{CUGold}{\textbf{Wednesday, September 28}}\\
We will share and discuss the findings from our laboratory assignments.

    \subsection{Assignments}
    \begin{description}%[itemsep=2pt,labelsep=0pt]
        \item[Lab Assignment 1 ] -- due September 28. 
    \end{description}

%%%%%%%%%%%%%
%% WEEK 07 %%
%%%%%%%%%%%%%
\section{12 -- Networks}
\textcolor{CUGold}{\textbf{Monday, October 3}}\\

\section{13 -- Networks Lab}
\textcolor{CUGold}{\textbf{Wednesday, October 5}}\\
We will share and discuss the findings from our laboratory assignments.

    \subsection{Assignments}
    \begin{description}%[itemsep=2pt,labelsep=0pt]
        \item[Lab Assignment 2 ] -- due Wednesday, October 5. 
    \end{description}

%%%%%%%%%%%%%
%% WEEK 08 %%
%%%%%%%%%%%%%
\section{14 -- Pageviews}
\textcolor{CUGold}{\textbf{Monday, October 10}}\\
Wikipedia collects data about the volume of pageviews for each article. What do these patterns of information consumption reveal? How can they be combined with other kinds of data to measure inefficiencies or make forecasts?

    \subsection{Readings}
    \begin{readinglist}
        \item \bibentry{mestyan_early_2013}
        \item \bibentry{warncke-wang_misalignment_2015}
    \end{readinglist}
    
    \subsection{Additional Readings}
    \begin{readinglist}
        \item \bibentry{generous_global_2014}
        \item \bibentry{hickmann_forecasting_2015}
    \end{readinglist}

\section{15 -- Pageviews Lab}
\textcolor{CUGold}{\textbf{Wednesday, October 12}}\\
We will share and discuss the findings from our laboratory assignments.

    \subsection{Assignments}
    \begin{description}%[itemsep=2pt,labelsep=0pt]
        \item[Lab Assignment 3 ] -- due Wednesday, October 12. 
    \end{description}
    
%%%%%%%%%%%%%
%% WEEK 09 %%
%%%%%%%%%%%%%
\section{16 -- Ontologies}
\textcolor{CUGold}{\textbf{Monday, October 17}}\\
% Wikidata?

\section{17 -- Ontologies Lab}
\textcolor{CUGold}{\textbf{Wednesday, October 19}}\\
We will share and discuss the findings from our laboratory assignments.

    \subsection{Assignments}
    \begin{description}%[itemsep=2pt,labelsep=0pt]
        \item[Lab Assignment 4 ] -- due Wednesday, October 19. 
    \end{description}
    
%%%%%%%%%%%%%
%% WEEK 10 %%
%%%%%%%%%%%%%
\section{18 -- Bots}
\textcolor{CUGold}{\textbf{Monday, October 24}}\\

    \subsection{Readings}
    \begin{readinglist}
        \item \bibentry{geiger_work_2010}
        \item \bibentry{geiger_when_2013}
    \end{readinglist}
    
    \subsection{Additional Readings}
    \begin{readinglist}
        \item \bibentry{halfaker_snuggle_2014}
        \item \bibentry{steiner_bots_2014}
        \item \bibentry{muller-birn_work_2013}
    \end{readinglist}

\section{19 -- Bots Lab}
\textcolor{CUGold}{\textbf{Wednesday, October 26}}\\
We will share and discuss the findings from our laboratory assignments.

    \subsection{Assignments}
    \begin{description}%[itemsep=2pt,labelsep=0pt]
        \item[Lab Assignment 5 ] -- due Wednesday, October 26. 
    \end{description}
    
%%%%%%%%%%%%%
%% WEEK 11 %%
%%%%%%%%%%%%%
\section{20 -- Edge Case 1: High-Tempo Collaboration}
\textcolor{CUGold}{\textbf{Monday, October 31}}\\
Wikipedia's social and technical structures were organized assuming that its knowledge would be established and stable. However, Wikipedians have demonstrated a remarkable capacity to rapidly create and revise articles in response to current and breaking news events. 

    \subsection{Readings}
    \begin{readinglist}
        \item \bibentry{keegan_hot_2011}
        \item \bibentry{keegan_history_2013}
    \end{readinglist}
    
    \subsection{Additional Readings}
    \begin{readinglist}
        %\item \bibentry{bechky_gaffers_2006}
        \item \bibentry{bechky_expecting_2011}
        \item \bibentry{bigley_incident_2001}
        \item \bibentry{faraj_coordination_2006}
        \item \bibentry{keegan_is_2015}
        \item \bibentry{klein_dynamic_2006}
        \item \bibentry{majchrzak_coordinating_2007}
        \item \bibentry{palen_emergence_2008}
        \item \bibentry{palen_crisis_2016}
        %\item \bibentry{weick_collective_1993}
        \item \bibentry{weick_organizing_1999}
    \end{readinglist}

\section{21 -- Edge Case 2: Alternative Internet Encyclopedias}
\textcolor{CUGold}{\textbf{Wednesday, November 2}}\\
The MediaWiki software and encyclopedic mission of Wikipedia were adopted by other online communities that have very different norms and values. We will discuss how these alternative norms and values influenced the success and sustainability of these encyclopedias.

    \subsection{Readings}
    \textbf{\textit{Note}: These alternative encyclopedias contain content that you may find offensive.}\vspace{-10pt}
    \begin{readinglist}
        \item \bibentry{colbert_schlafly_2009}
        \item \bibentry{sanger_why_2007}
        \item \bibentry{creationwiki_about}
        \item \bibentry{rationalwiki_about}
        \item \bibentry{conservapedia_about}
        \item \bibentry{dramatica_about}
    \end{readinglist}
    
    \subsection{Assignments}
    \begin{description}%[itemsep=2pt,labelsep=0pt]
        \item[Reading Response 5 ] -- due Wednesday, November 2. What shortcomings did the creators of these alternative encyclopedias identify with Wikipedia's model? How did their
    \end{description}
    
    \subsection{Additional Readings}
    \begin{readinglist}
        \item \bibentry{lam_past_2011}
        \item \bibentry{hill_almost_2013}
        \item \bibentry{oneil_shirky_2010}
        \item \bibentry{sanger_fate_2009}
        \item \bibentry{shirky_larry_2006}
        \item \bibentry{sundin_debating_2007}
    \end{readinglist}
    
%%%%%%%%%%%%%
%% WEEK 12 %%
%%%%%%%%%%%%%
\section{22 -- Successors 1: Wikia}
\textcolor{CUGold}{\textbf{Monday, November 7}}

\section{23 -- Successors 2: OpenStreetMaps}
\textcolor{CUGold}{\textbf{Wednesday, November 9}}

    
    \subsection{Assignments}
    \begin{description}%[itemsep=2pt,labelsep=0pt]
        \item[Reading Response 6 ] -- due Wednesday, November 9. 
    \end{description}
    
%%%%%%%%%%%%%
%% WEEK 13 %%
%%%%%%%%%%%%%
\section{24 -- Creative Collaboration 1: Scratch}
\textcolor{CUGold}{\textbf{Monday, November 14}}

    \subsection{Readings}
    \begin{readinglist}
        \item \bibentry{resnick_scratch_2009}
        \item \bibentry{roque_supporting_2016}
    \end{readinglist}
    
    \subsection{Additional Readings}
    \begin{readinglist}
        \item \bibentry{dasgupta_remixing_2016}
        \item \bibentry{hill_cost_2013}
        \item \bibentry{monroy-hernandez_computers_2011}
        \item \bibentry{nickerson_appropriation_2011}
    \end{readinglist}

\section{25 -- Creative Collaboration 2: Threadless}
\textcolor{CUGold}{\textbf{Wednesday, November 16}}

\begin{readinglist}
    \item \bibentry{brabham_moving_2010}
    \item \bibentry{hutter_communitition_2011}
\end{readinglist}
    
    \subsection{Assignments}
    \begin{description}%[itemsep=2pt,labelsep=0pt]
        \item[Reading Response 7 ] -- due Wednesday, November 16. 
    \end{description}
    
%%%%%%%%%%%%%
%% WEEK 14 %%
%%%%%%%%%%%%%
\section{26 -- Fall Break}
\textcolor{CUGold}{\textbf{Monday, November 21  --- NO CLASS}}

\section{27 -- Fall Break}
\textcolor{CUGold}{\textbf{Wednesday, November 23 --- NO CLASS}}

%%%%%%%%%%%%%
%% WEEK 15 %%
%%%%%%%%%%%%%
\section{28 -- Beyond Wikis: Open Hardware}
\textcolor{CUGold}{\textbf{Monday, November 28}}

    \subsection{Readings}
    \begin{readinglist}
        \item \bibentry{buechley_lilypad_2010}
        \item \bibentry{kuznetsov_rise_2010}
        \item \bibentry{weiss_open_2008}
    \end{readinglist}
    
    \subsection{Additional Readings}
    \begin{readinglist}
        \item \bibentry{buechley_diy_2009}
        \item \bibentry{malinen_open_2010}
        \item \bibentry{mellis_fab_2011}
        \item \bibentry{mellis_collaboration_2012}
    \end{readinglist}

\section{29 -- Beyond Wikis: Blockchains}
\textcolor{CUGold}{\textbf{Wednesday, November 30}}

    \subsection{Readings}
    \begin{readinglist}
        \item \bibentry{swan_blockchain_2015}. Skim Chapters 1--3.
    \end{readinglist}
    
    \subsection{Assignments}
    \begin{description}%[itemsep=2pt,labelsep=0pt]
        \item[Reading Response 8 ] -- due Wednesday, November 30. 
    \end{description}
    
%%%%%%%%%%%%%
%% WEEK 16 %%
%%%%%%%%%%%%%
\section{30 -- Platform Cooperativism}
\textcolor{CUGold}{\textbf{Monday, December 5}}

    \subsection{Readings}
    \begin{readinglist}
        \item \bibentry{louis_user_2013}
        \item \bibentry{scholz_platform_2014}
        %\item \bibentry{schneider_internet_2015}
    \end{readinglist}

\section{31 -- Wrap-up}
\textcolor{CUGold}{\textbf{Wednesday, December 7}}



%%%%%%%%%%%%%%%%%%%%%%%%%
%%%%%%%%%%%%%%%%%%%%%%%%%
%%%% END OF SEMESTER %%%%
%%%%%%%%%%%%%%%%%%%%%%%%%
%%%%%%%%%%%%%%%%%%%%%%%%%

% bibliography here
\newpage
\renewcommand{\bibsection}{\section{\huge \bibname}\prebibhook}
\baselineskip 14.2pt
\bibliography{refs}
\bibliographystyle{apalike}

\end{document}
